\documentclass{article}
\usepackage{graphicx}
\title{Мини-конспект по теме: Теорема Пифагора}
\author{Игорь Ерёменко}
\date{16 июля 2025 г.}
\usepackage{hyperref}
\usepackage[russian]{babel}
\usepackage{graphicx}
\begin{document}
\maketitle
\tableofcontents
\newpage
\section{Введение}
{Теорема Пифагора — одна из важнейших теорем евклидовой геометрии. Она находит
применение в самых разных областях:}
\begin{itemize}
    \item геометрия и тригонометрия
    \item физика
    \item инженерные расчеты
    \item компьютерная графика
\end{itemize}
\section{Формулирование теоремы}
\textbf{Слова:}
{В прямоугольном треугольнике квадрат гипотенузы равен сумме квадратов
катетов.}
\newline
\begin{equation}
c^2 = a^2 + b^2
\end{equation}
\newline
\hspace{8mm} 
Как видно из формулы 1, знание двух сторон позволяет найти третью.
\section{Доказательство(набросок)}
\begin{center}
Одно из доказательств основывается на площади квадрата, составленного
из четырёх одинаковых прямоугольных треугольников и малого квадрата
в центре. Раскладывая площадь двумя способами, получаем $c^2 = a^2 + b^2$

\end{center}
\section{Пример рассчета}
\textbf{Пример 1}
\begin{center}
a = 3, b = 4
\end{center}
\begin{equation}
c = \sqrt{a^2 + b^2} = \sqrt {9+ 16} = 5
\end{equation}
\textbf{Пример 2}
\begin{enumerate}
    \item Дано:
    {a = 5, b = 12}
    \item Решение:
    \begin{equation}
c = \sqrt{5^2 + 12^2} = \sqrt {25+ 144} = 13
\end{equation}
\end{enumerate}
\section{Таблица значений}
\hspace{3cm}
\begin{tabular}{|c|c|c|}
\hline
Катет а & Катет b & Гипотенуза с \\
\hline
3 & 4 & 5 \\
\hline
5 & 12 & 13 \\
\hline
7 & 24 & 25 \\
\hline
\end{tabular}
\section{Иллюстрация}
Ниже пример изображения (не забудьте добавить файл triangle.png в ту же папку):
\begin{center}
\newline
\includegraphics[width=0.5\textwidth]{triangle.png}
\end{center}
\section{Заключение}
Теорема Пифагора — один из краеугольных камней геометрии, помогающий решать
множество практических задач.
\section{Ссылки и литература}
\begin{itemize}
\item \href{https://ru.wikipedia.org/wiki/Теорема_Пифагора}{Википедия: Теорема Пифогора}
\item Классические учебники геометрии
\end{itemize}
\end{document}